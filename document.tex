% Template for PLoS
% Version 1.0 January 2009
%
% To compile to pdf, run:
% latex plos.template
% bibtex plos.template
% latex plos.template
% latex plos.template
% dvipdf plos.template

\documentclass[10pt]{article}

% amsmath package, useful for mathematical formulas
\usepackage{amsmath}
% amssymb package, useful for mathematical symbols
\usepackage{amssymb}

% graphicx package, useful for including eps and pdf graphics
% include graphics with the command \includegraphics
\usepackage{graphicx}

% cite package, to clean up citations in the main text. Do not remove.
\usepackage{cite}

\usepackage{color} 

% Use doublespacing - comment out for single spacing
%\usepackage{setspace} 
%\doublespacing

% Text layout
\topmargin 0.0cm
\oddsidemargin 0.5cm
\evensidemargin 0.5cm
\textwidth 16cm 
\textheight 21cm

% Bold the 'Figure #' in the caption and separate it with a period
% Captions will be left justified
\usepackage[labelfont=bf,labelsep=period,justification=raggedright]{caption}

% Use the PLoS provided bibtex style
\bibliographystyle{plos2009}

% Remove brackets from numbering in List of References
\makeatletter
\renewcommand{\@biblabel}[1]{\quad#1.}
\makeatother


% Leave date blank
\date{}

\pagestyle{myheadings}
%% ** EDIT HERE **


%% ** EDIT HERE **
%% PLEASE INCLUDE ALL MACROS BELOW

%% END MACROS SECTION

\begin{document}

% Title must be 150 characters or less
\begin{flushleft}
{\Large
\textbf{The Functional Therapeutic Chemical Classification System}
}
% Insert Author names, affiliations and corresponding author email.
\\
Samuel Croset$^{1,\ast}$, 
John Overington$^{2}$, 
Dietrich Rebholz-Schuhmann$^{3}$
\\
\bf{1} Samuel Croset Dept/Program/Center, Institution Name, City, State, Country
\\
\bf{2} John Overington Dept/Program/Center, Institution Name, City, State, Country
\\
\bf{3} Dietrich Rebholz-Schuhmann Dept/Program/Center, Institution Name, City, State, Country
\\
$\ast$ E-mail: Corresponding croset@ebi.ac.uk
\end{flushleft}

% Please keep the abstract between 250 and 300 words
\section*{Abstract}

The abstract of the paper should be succinct; it must not exceed 300 words. 
Authors should mention the techniques used without going into methodological detail 
and should summarize the most important results. While the abstract is conceptually divided into 
three sections (Background, Methodology/Principal Findings, and Conclusions/Significance), please 
do not apply these distinct headings to the abstract within the article file. Please do 
not include any citations and avoid specialist abbreviations.


% Please keep the Author Summary between 150 and 200 words
% Use first person. PLoS ONE authors please skip this step. 
% Author Summary not valid for PLoS ONE submissions.   
\section*{Author Summary}

We ask that all authors of research articles include a 150–200 word non-technical summary 
of the work as part of the manuscript to immediately follow the abstract. This text is subject to editorial 
change, should be written in the first-person voice, and should be distinct from the scientific abstract. Aim to 
highlight where your work fits within a broader context; present the significance or possible implications of your work 
simply and objectively; and avoid the use of acronyms and complex terminology wherever possible. The goal is to make your 
findings accessible to a wide audience that includes both scientists and non-scientists. Authors may benefit from consulting 
with a science writer or press officer to ensure they effectively communicate their findings to a general audience. Examples 
are available at:

\section*{Introduction}

The introduction \cite{Krotzsch2012} should put the focus of the manuscript into a broader context. 
As you compose the introduction, think of readers who are not experts in this field. 
Include a brief review of the key literature. If there are relevant controversies or disagreements in 
the field, they should be mentioned so that a non-expert reader can delve into these issues further. 
The introduction should conclude with a brief statement of the overall aim of the experiments and a comment 
about whether that aim was achieved.

Functional Therapeutic Chemical Classification System (FTC) is a taxonomy (ontology) representing the mode of action of approved
drugs in the human body.
It can be pictured as a toolbox classifying the therapeutic agents based on their functionality in the human body.
FTC categories are automatically built based on Gene Ontology terms.
FTC categories are also equivalent to arbitrary definitions such as drug that perturbs some (protein that involved-in some (regulation of 
biological process)).
A knowledge base created by integrating the content of DrugBank and some GO annotations assists to automatically
assign compounds into FTC categories based on the satisfaction of the definition.
The FTC is evaluated against the Anatomical Therapeutic Chemical Classification System by assigning equivalences between classes.
The knowledge base can be queried in order to retrieve implicit information about the involvement of approved drugs into biological
processes.
Drug repurposing hypotheses can be formulated out of the evaluation.
The FTC can serve as starting point for analysis or projects requiring a structured vocabulary representing the mode of action.
The source code of the FTC as well as the data are publicly available for free to anyone.

The indication of a drug depends on the effects induced by the molecule when administrated in a biological system.
For instance, a molecule such as the Ximelagatran is indicated as anticoagulant because
it directly inhibits the catalytic activity of the thrombin, 
enzyme itself involved in the blood coagulation.
In order to speak of this therapeutic effect (anticoagulant), people sometimes refer to 
the **mode of action** of an active compound. 
Wikipedia gives us the following definition for it:
> A mode of action describes a functional or anatomical change, at the 
> cellular level, resulting from the exposure of a living organism to a substance. 

Characterising the mode of action of drugs is interesting for the following reasons:
It is a central link between the molecular structure of a drug and its indication towards a disease. Drugs with similar
mode of actions could be used to treat similar conditions.
It abstracts away from the chemical structure and allows to reason over the effects of the compound in the body. The mode of action
of a drug is a feature against which compounds can be classified.
It can captures polypharmacology. Most of the drugs are not magic bullets
and produce various side effects. A drug can have several mode of actions, representing the different processes the drug is perturbing directly
or indirectly.
A representation of the mode of action helps to better understand and analyse the behaviour of an agent in the human body.

Although the **mode of action** is an old and useful notion in the drug discovery 
community (see number of results for thisPubMed search, no resources were
previously fully dedicated to it. The aim of the FTC is to fill this gap.

The FTC can be seen as a toolbox designed to fix dysfunctioning biological systems.
It features over 20'000 compartments (categories) inside which approved 
drugs are automatically classified, based on facts from various databases.
FTC categories represent the mode of action of compounds in the human system. 
The resource can be used for various tasks where a structured vocabulary is needed, such as compound annotation, drug repurposing analysis and 
anything else you can think of!


% Results and Discussion can be combined.
\section*{Results}

The results section should provide details of all of the experiments that 
are required to support the conclusions of the paper. There is no specific word limit for 
this section, but details of experiments that are peripheral to the main thrust of the article and that 
detract from the focus of the article should not be included. The section may be divided into subsections, 
each with a concise subheading. Large datasets, including raw data, should be submitted as supporting files; 
these are published online alongside the accepted article. The results section should be written in the past tense.

\subsection*{Subsection 1}

\subsection*{Subsection 2}

\section*{Discussion}

The discussion should spell out the major conclusions of the work along with some 
explanation or speculation on the significance of these conclusions. How do the 
conclusions affect the existing assumptions and models in the field? How can future research 
build on these observations? What are the key experiments that must be done? The discussion should 
be concise and tightly argued. The results and discussion may be combined into one section, if desired.

% You may title this section "Methods" or "Models". 
% "Models" is not a valid title for PLoS ONE authors. However, PLoS ONE
% authors may use "Analysis" 
\section*{Materials and Methods}

This section should provide enough detail for reproduction of the findings. Protocols 
for new methods should be included, but well-established protocols may simply be referenced. While we do 
encourage authors to submit all appendices, detailed protocols, or details of the algorithms for newer or less 
well-established methods, please do so as Supporting Information files. These are not included in the typeset manuscript, 
but are downloadable and fully searchable from the HTML version of the article.

% Do NOT remove this, even if you are not including acknowledgments
\section*{Acknowledgments}

People who contributed to the work but do not fit the criteria for authors should be listed in the 
Acknowledgments, along with their contributions. You must also ensure that anyone named in the Acknowledgments 
agrees to being so named.

Details of the funding sources that have supported the work should be confined to the funding statement 
provided in the online submission system. Do not include them in the Acknowledgments.

%\section*{References}
% The bibtex filename
\bibliography{document}

\section*{Figure Legends}
%\begin{figure}[!ht]
%\begin{center}
%%\includegraphics[width=4in]{figure_name.2.eps}
%\end{center}
%\caption{
%{\bf Bold the first sentence.}  Rest of figure 2  caption.  Caption 
%should be left justified, as specified by the options to the caption 
%package.
%}
%\label{Figure_label}
%\end{figure}

The aim of the figure legend should be to describe the key messages of the figure, but 
the figure should also be discussed in the text. An enlarged version of the figure and its full legend 
will often be viewed in a separate window online, and it should be possible for a reader to understand the figure 
without switching back and forth between this window and the relevant parts of the text. Each legend should have 
a concise title of no more than 15 words that can stand alone, without the use of figure part labels. The overall 
legend itself should be succinct, while still explaining all figure parts, symbols and abbreviations. Avoid lengthy descriptions of methods.

\section*{Tables}
%\begin{table}[!ht]
%\caption{
%\bf{Table title}}
%\begin{tabular}{|c|c|c|}
%table information
%\end{tabular}
%\begin{flushleft}Table caption
%\end{flushleft}
%\label{tab:label}
% \end{table}

Tables should be included at the end of the manuscript file and cited sequentially in the text. 
All tables should have a concise title. Footnotes can be used to explain abbreviations. 
Citations should be indicated using the same style as outlined above. Tables should not occupy more than one 
printed page; larger tables can be published as online supporting information. Tables must be cell-based; 
do not use picture elements, text boxes, tabs, or returns in tables. Please ensure that all tables conform to our 
Guidelines for Figure and Table Preparation when preparing them.

\end{document}

